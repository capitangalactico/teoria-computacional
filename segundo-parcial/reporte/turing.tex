\section{Maquina de Turing}
	\subsection{Descripción del problema}
	Cosas chidas
	\begin{figure}[H]
		\begin{center}
		\includegraphics[width=14cm, height=7cm]{img/turing.png}
		\caption{Representación de una Maquina de Turing}
		\label{fig:diagrama-turing}
		\end{center}
	\end{figure}
	\subsection{Código}
	El código fue realizado en Python 3.5.
	\\Archivo: main\_turing.py
	\begin{lstlisting}[language=Python]
	print('main_turing.py')
	\end{lstlisting}
	\\Archivo: maquina\_turing.py
	\begin{lstlisting}[language=Python]
	print('maquina de turing.py')
	\end{lstlisting}
	\subsection{Pruebas}
	Pruebas de las opciones del menú.
	\\
	{\large Modo de consola.}
	\begin{figure}[H]
		\begin{center}
			\includegraphics[width=\linewidth, height=6cm]{img/turing-manual.png}
			\caption{Historia de la Maquina de Turing.}
			\label{fig:turing1}
		\end{center}
	\end{figure}
	{\large Modo archivo.}
	\begin{figure}[H]
		\begin{center}
			\includegraphics[width=\linewidth, height=20cm]{img/turing-automatico.png}
			\caption{Parte de la historia de la Maquina de Turing.}
			\label{fig:turing2}
		\end{center}
	\end{figure}